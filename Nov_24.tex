\documentclass[11pt,letterpaper,boxed]{hmcpset}
\usepackage[margin=1in]{geometry}
\usepackage{graphicx}
\usepackage{enumerate}
\usepackage{amsmath}
\usepackage{mathtools}
\usepackage{amssymb}
\usepackage{cancel}

\setlength{\parskip}{6pt}
\setlength{\parindent}{0pt}

% convenient delimiters
\newcommand{\set}[1]{\ensuremath{ \left\{ #1 \right\} }}
\newcommand{\pn}[1]{\left( #1 \right)}
\newcommand{\abs}[1]{\left| #1 \right|}
\newcommand{\bk}[1]{\left[ #1 \right]}
\newcommand{\vc}[1]{\left\langle #1 \right\rangle}

% set numbering style for enumerated lists to be of form (a), (b), (c), etc.
\renewcommand{\labelenumi}{{(\alph{enumi})}}


\name{}
\class{Math 35, Section -  }
\assignment{Problem Set 5}
\duedate{November 24, 2015}


\begin{document} {

    \problemlist{7.\{1.1, 1.3, 1.5, 2.20, 3.30abc, 3.33c, 3.37a\}, 8.\{ 1.4abc, 2.21ab, 2.25, 4.44ab, 4.45\}}

%%%%%%%%%%%%%%% Number 1 %%%%%%%%%%%%%%%%%%%%%%%%%%%%%%%%%%%%%%%%%%%%%%%%%

\begin{problem}[7.1.1]
	Consider a normal population distribution with the value of $\sigma$ known.
	\begin{enumerate}
		\item
			What is the confidence level for the interval $\overline{x} \pm 2.81 \sigma / \sqrt{n}$?
		\item
			What is the confidence level for the interval $\overline{x} \pm 1.44 \sigma / \sqrt{n}$?
		\item
			What value of $z_{\alpha/2}$ in the CI formula (7.5) results in a confidence level of 99.7$\%$?
		\item
			Answer the question posed in part (c) for a confidence level of 75$\%$.
	\end{enumerate}
\end{problem}

\begin{solution}
	\vfill
\end{solution}
\newpage

%%%%%%%%%%%%%%% Number 2 %%%%%%%%%%%%%%%%%%%%%%%%%%%%%%%%%%%%%%%%%%%%%%%%%

\begin{problem}[7.1.3]
	Suppose that a random sample of 50 bottles of a particular brand of cough syrup is selected and the alcohol content of each bottle is determined. Let $\mu$ denote the average alcohol content for the population of all bottles of the brand under study. Suppose that the resulting 95$\%$ confidence interval is (7.8, 9.4).
	\begin{enumerate}
		\item
			Would a 90$\%$ confidence interval calculated from this same sample have been narrower or wider than the given interval? Explain your reasoning.
		\item
			Consider the following statement: There is a 95$\%$ chance that $\mu$ is between 7.8 and 9.4. Is this statement correct? Why or why not?
		\item
			Consider the following statement: We can be highly confident that 95$\%$ of all bottles of this type of cough syrup have an alcohol content that is between 7.8 and 9.4. Is this statement correct? Why or why not?
		\item
			Consider the following statement: If the process of selecting a sample of size 50 and then computing the corresponding 95$\%$ interval is repeated 100 times, 95 of the resulting intervals will include $\mu$ . Is this statement correct? Why or why not?
	\end{enumerate}
\end{problem}

\begin{solution}
	\vfill
	\end{solution}
\newpage

%%%%%%%%%%%%%%% Number 3 %%%%%%%%%%%%%%%%%%%%%%%%%%%%%%%%%%%%%%%%%%%%%%%%%

\begin{problem}[7.1.5]
	Assume that the helium porosity (in percentage) of coal samples taken from any particular seam is normally distributed with true standard deviation .75.
	\begin{enumerate}
		\item
			Compute a 95$\%$ CI for the true average porosity of a certain seam if the average porosity for 20 specimens from the seam was 4.85.
		\item
			Compute a 98$\%$ CI for true average porosity of another seam based on 16 specimens with a sample average porosity of 4.56.
		\item
			How large a sample size is necessary if the width of the 95$\%$ interval is to be .40?
		\item
			What sample size is necessary to estimate true average porosity to within .2 with 99$\%$ confidence?
	\end{enumerate}
\end{problem}

\begin{solution}
	\vfill
\end{solution}
\newpage

%%%%%%%%%%%%%%% Number 4 %%%%%%%%%%%%%%%%%%%%%%%%%%%%%%%%%%%%%%%%%%%%%%%%%

\begin{problem}[7.2.20]
	TV advertising agencies face increasing challenges in reaching audience members because viewing TV programs via digital streaming is gaining in popularity. The \textbf{Harris poll} reported on November 13, 2012, that 53$\%$ of 2343 American adults surveyed said they have watched digitally streamed TV programming on some type of device.
	\begin{enumerate}
		\item
			Calculate and interpret a confidence interval at the 99$\%$ confidence level for the proportion of all adult Americans who watched streamed programming up to that point in time.
		\item
			What sample size would be required for the width of a 99$\%$ CI to be at most .05 irrespective of the value of $\hat{p}$?
	\end{enumerate}
\end{problem}

\begin{solution}
	\vfill
\end{solution}
\newpage

%%%%%%%%%%%%%%% Number 5 %%%%%%%%%%%%%%%%%%%%%%%%%%%%%%%%%%%%%%%%%%%%%%%%%

\begin{problem}[7.3.30abc]
	Determine the $t$ critical value for a two-sided confidence interval in each of the following situations:
	\begin{enumerate}
		\item
			Confidence level = 95$\%$, df = 10
		\item
			Confidence level = 95$\%$, df = 15
		\item
			Confidence level = 99$\%$, df = 10
	\end{enumerate}
\end{problem}

\begin{solution}
	\vfill
\end{solution}
\newpage

%%%%%%%%%%%%%%% Number 6 %%%%%%%%%%%%%%%%%%%%%%%%%%%%%%%%%%%%%%%%%%%%%%%%%

\begin{problem}[7.3.33c]
	The article \textbf{"Measuring and Understanding the Aging of Kraft Insulating Paper in Power Transformers" (\emph{IEEE Electrical Insul. Mag.}, 1996: 28--34)} contained the following observations on degree of polymerization for paper specimens for which viscosity times concentration fell in a certain middle range:\\
	\begin{center}
		\begin{tabular}{c c c c c c c}
 			418 & 421 & 421 & 434 & 437 & 439 & 454\\
			463 & 465 & 422 & 425 & 427 & 431 & 446\\
			447 & 448 & 453\\
		 \end{tabular}
	\end{center}
\qquad (c) Calculate a two-sided 95$\%$ confidence interval for true average degree of polymerization (as did the authors of the article). Does the interval suggest that 440 is a plausible value for true average degree of polymerization? What about 450?
\end{problem}

\begin{solution}
	\vfill
\end{solution}
\newpage
%%%%%%%%%%%%%%% Number 7 %%%%%%%%%%%%%%%%%%%%%%%%%%%%%%%%%%%%%%%%%%%%%%%%%

\begin{problem}[7.3.37a]
	TA study of the ability of individuals to walk in a straight line \textbf{("Can We Really Walk Straight?" \emph{Amer. J. of Physical Anthro.}, 1992: 19?27)} reported the accompanying data on cadence (strides per second) for a sample of n 5 20 randomly selected healthy men.\\
	\begin{center}
		\begin{tabular}{c c c c c c c c c c}
 			.95 & .85 & .92 & .95 & .93 & .86 & 1.00 & .92 & .85 & .81\\
			.78 & .93 & .93 & 1.05 & .93 & 1.06 & 1.06 & .96 & .81 & .96\\
		 \end{tabular}
	\end{center}
A normal probability plot gives substantial support to the assumption that the population distribution of cadence is approximately normal. A descriptive summary of the data from Minitab follows:\\
	\begin{center}
		\begin{tabular}{c | c c c c c c}
 			Variable & N & Mean & Median & TrMean & StDev & SEMean\\
			cadence & 20 & 0.9255 & 0.9300 & 0.9261 & 0.0809 & 0.0181\\
			\hline
			Variable & & Min & Max & Q1 & Q3\\
			cadence & & 0.7800 & 1.0600 & 0.8525 & 0.9600\\
		 \end{tabular}
	\end{center}
	\begin{enumerate}
		\item
			Calculate and interpret a 95$\%$ confidence interval for population mean cadence.
	\end{enumerate}
\end{problem}
\begin{solution}
	\vfill
\end{solution}
\newpage

%%%%%%%%%%%%%%% Number 8 %%%%%%%%%%%%%%%%%%%%%%%%%%%%%%%%%%%%%%%%%%%%%%%%%

\begin{problem}[8.1.4abc]
	Pairs of $P$-values and significance levels, $\alpha$ , are given. For each pair, state whether the observed $P$-value would lead to rejection of $H_0$ at the given significance level.\\
	\begin{enumerate}
		\item
			$P$-value = .084, $\alpha = .05$
		\item
			$P$-value = .003, $\alpha = .001$
		\item
			$P$-value = .498, $\alpha = .05$
	\end{enumerate}
\end{problem}
\begin{solution}
	\vfill
\end{solution}
\newpage

%%%%%%%%%%%%%%% Number 9 %%%%%%%%%%%%%%%%%%%%%%%%%%%%%%%%%%%%%%%%%%%%%%%%%

\begin{problem}[8.2.21ab]
	 The desired percentage of SiO$_2$ in a certain type of aluminous cement is 5.5. To test whether the true average percentage is 5.5 for a particular production facility, 16 independently obtained samples are analyzed. Suppose that the percentage of SiO$_2$ in a sample is normally distributed with $\sigma = .3$ and that $\overline{x} = 5.25$.\\
	\begin{enumerate}
		\item
			Does this indicate conclusively that the true average percentage differs from 5.5?
		\item
			If the true average percentage is $\mu = 5.6$ and a level $\alpha = .01$ test based on $n = 16$ is used, what is the probability of detecting this departure from H$_0$?
	\end{enumerate}
\end{problem}
\begin{solution}
	\vfill
\end{solution}
\newpage

%%%%%%%%%%%%%%% Number 10 %%%%%%%%%%%%%%%%%%%%%%%%%%%%%%%%%%%%%%%%%%%%%%%%%

\begin{problem}[8.2.25]
	 Body armor provides critical protection for law enforcement personnel, but it does affect balance and mobility. The article \textbf{"Impact of Police Body Armour and Equipment on Mobility" (\emph{Applied Ergonomics}, 2013: 957--961)} reported that for a sample of 52 male enforcement officers who underwent an acceleration task that simulated exiting a vehicle while wearing armor, the sample mean was 1.95 sec, and the sample standard deviation was .20 sec. Does it appear that true average task time is less than 2 sec? Carry out a test of appropriate hypotheses using a significance level of .01.
\end{problem}
\begin{solution}
	\vfill
\end{solution}
\newpage

%%%%%%%%%%%%%%% Number 11 %%%%%%%%%%%%%%%%%%%%%%%%%%%%%%%%%%%%%%%%%%%%%%%%%

\begin{problem}[8.4.44ab]
	 A manufacturer of nickel-hydrogen batteries randomly selects 100 nickel plates for test cells, cycles them a specified number of times, and determines that 14 of the plates have blistered.
	 \begin{enumerate}
		 \item
		 	Does this provide compelling evidence for concluding that more than 10$\%$ of all plates blister under such circumstances? State and test the appropriate hypotheses using a significance level of .05. In reaching your conclusion, what type of error might you have committed?
		\item
			If it is really the case that 15$\%$ of all plates blister under these circumstances and a sample size of 100 is used, how likely is it that the null hypothesis of part (a) will not be rejected by the level .05 test? Answer this question for a sample size of 200.
	 \end{enumerate}
\end{problem}
\begin{solution}
	\vfill
\end{solution}
\newpage

%%%%%%%%%%%%%%% Number 12 %%%%%%%%%%%%%%%%%%%%%%%%%%%%%%%%%%%%%%%%%%%%%%%%%

\begin{problem}[8.4.45]
	 A random sample of 150 recent donations at a certain blood bank reveals that 82 were type A blood. Does this suggest that the actual percentage of type A donations differs from 40$\%$, the percentage of the population having type A blood? Carry out a test of the appropriate hypotheses using a significance level of .01. Would your conclusion have been different if a significance level of .05 had been used?
\end{problem}
\begin{solution}
	\vfill
\end{solution}

\end{document}
