\documentclass[11pt,letterpaper,boxed]{hmcpset}
\usepackage[margin=1in]{geometry}
\usepackage{graphicx}
\usepackage{enumerate}
\usepackage{amsmath}
\usepackage{mathtools}
\usepackage{amssymb}
\usepackage{cancel}

\setlength{\parskip}{6pt}
\setlength{\parindent}{0pt}

% convenient delimiters
\newcommand{\set}[1]{\ensuremath{ \left\{ #1 \right\} }}
\newcommand{\pn}[1]{\left( #1 \right)}
\newcommand{\abs}[1]{\left| #1 \right|}
\newcommand{\bk}[1]{\left[ #1 \right]}
\newcommand{\vc}[1]{\left\langle #1 \right\rangle}

% set numbering style for enumerated lists to be of form (a), (b), (c), etc.
\renewcommand{\labelenumi}{{(\alph{enumi})}}


\name{ }
\class{Math 35, Section -   }
\assignment{Problem Set 6}
\duedate{December 1, 2015}


\begin{document} {

    \problemlist{9.1.\{1, 3, 7, 8\}}

%%%%%%%%%%%%%%% Number 1 %%%%%%%%%%%%%%%%%%%%%%%%%%%%%%%%%%%%%%%%%%%%%%%%%

\begin{problem}[9.1.1]
	An article in the November 1983 \textbf{\emph{Consumer Reports}} compared various types of batteries. The average lifetimes of Duracell Alkaline AA batteries and Eveready Energizer Alkaline AA batteries were given as 4.1 hours and 4.5 hours, respectively. Suppose these are the population average lifetimes.
	\begin{enumerate}
		\item
			Let $\overline{X}$ be the sample average lifetime of 100 Duracell batteries and $\overline{Y}$ be the sample average lifetime of 100 Eveready batteries. What is the mean value of $\overline{X} - \overline{Y}$ (i.e., where is the distribution of $\overline{X} - \overline{Y}$ centered)? How does your answer depend on the specified sample sizes?
		\item
			Suppose the population standard deviations of lifetime are 1.8 hours for Duracell batteries and 2.0 hours for Eveready batteries. With the sample sizes given in part (a), what is the variance of the statistic $\overline{X} - \overline{Y}$, and what is its standard deviation?
		\item
			For the sample sizes given in part (a), draw a picture of the approximate distribution curve of $\overline{X} - \overline{Y}$ (include a measurement scale on the horizontal axis). Would the shape of the curve necessarily be the same for sample sizes of 10 batteries of each type? Explain.
	\end{enumerate}
\end{problem}

\begin{solution}
	\vfill
\end{solution}
\newpage

%%%%%%%%%%%%%%% Number 2 %%%%%%%%%%%%%%%%%%%%%%%%%%%%%%%%%%%%%%%%%%%%%%%%%

\begin{problem}[9.1.3]
	Pilates is a popular set of exercises for the treatment of individuals with lower back pain. The method has six basic principles: centering, concentration, control, precision, flow, and breathing. The article \textbf{"Efficacy of the Addition of Modified Pilates Exercises to a Minimal Intervention in Patients with Chronic Low Back Pain: A Randomized Controlled Trial" \emph{(Physical Therapy}, 2013: 309--321)} reported on an experiment involving 86 subjects with nonspecific low back pain. The participants were randomly divided into two groups of equal size. The first group received just educational materials, whereas the second group participated in 6 weeks of Pilates exercises. The sample mean level of pain (on a scale from 0 to 10) for the control group at a 6-week follow-up was 5.2 and the sample mean for the treatment group was 3.1; both sample standard deviations were 2.3.
	\begin{enumerate}
		\item
			Does it appear that true average pain level for the control condition exceeds that for the treatment condition? Carry out a test of hypotheses using a significance level of .01 (the cited article reported statistical significance at this $\alpha$, and a sample mean difference of 2.1 also suggests practical significance).
		\item
			Does it appear that true average pain level for the control condition exceeds that for the treatment condition by more than 1? Carry out a test of appropriate hypotheses.
	\end{enumerate}
\end{problem}

\begin{solution}
	\vfill
\end{solution}
\newpage

%%%%%%%%%%%%%%% Number 3 %%%%%%%%%%%%%%%%%%%%%%%%%%%%%%%%%%%%%%%%%%%%%%%%%

\begin{problem}[9.1.7]
	Is there any systematic tendency for part-time college faculty to hold their students to different standards than do full-time faculty? The article \textbf{"Are There Instructional Differences Between Full-Time and Part-Time Faculty?" (\emph{College Teaching}, 2009: 23--26)} reported that for a sample of 125 courses taught by full- time faculty, the mean course GPA was 2.7186 and the standard deviation was .63342, whereas for a sample of 88 courses taught by part-timers, the mean and standard deviation were 2.8639 and .49241, respectively. Does it appear that true average course GPA for part-time faculty differs from that for faculty teaching full-time? Test the appropriate hypotheses at significance level .01.
\end{problem}

\begin{solution}
	\vfill
\end{solution}
\newpage

%%%%%%%%%%%%%%% Number 4 %%%%%%%%%%%%%%%%%%%%%%%%%%%%%%%%%%%%%%%%%%%%%%%%%

\begin{problem}[9.1.8]
	Tensile-strength tests were carried out on two different grades of wire rod \textbf{("Fluidized Bed Patenting of Wire Rods," \emph{Wire J.}, June 1977: 56--61)}, resulting in the accompanying data.
	\begin{center}
	\begin{tabular}{c c c c}
		Grade & Sample Size & Sample Mean (kg/mm$^2$) & Sample SD\\
		\hline
		AISI 1064 & $m = 129$ & $\overline{x} = 107.6$ & $s_1 = 1.3$\\
		AISI 1078 & $n = 129$ & $\overline{y} = 123.6$ & $s_2 = 2.0$\\
		\hline
	\end{tabular}
\end{center}
	\begin{enumerate}
		\item
			 Does the data provide compelling evidence for concluding that true average strength for the 1078 grade exceeds that for the 1064 grade by more than 10 kg/mm$^2$? Test the appropriate hypotheses using a significance level of .01.
		\item
			Estimate the difference between true average strengths for the two grades in a way that provides information about precision and reliability. (\textit{Compute 99\% CI is what Sakai says for this part of the problem}).
	\end{enumerate}
\end{problem}

\begin{solution}
	\vfill
\end{solution}

\end{document}
